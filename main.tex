\documentclass{article}
\usepackage[utf8]{inputenc}
\usepackage[spanish]{babel}
\usepackage{listings}
\usepackage{graphicx}
\graphicspath{ {images/} }
\usepackage{cite}

\begin{document}

\begin{titlepage}
    \begin{center}
        \vspace*{1cm}
            
        \Huge
        \textbf{Actividad N°1}
            
        \vspace{0.5cm}
        \LARGE
        Parcial 1 - Calistenia (15 porciento)
            
        \vspace{1.5cm}
            
        \textbf{Johan David Rojas Martinez}
            
        \vfill
            
        \vspace{0.8cm}
       
        \Large
\begin{figure}[h]
\includegraphics[width=4cm]{logoudea.png}
\centering
\end{figure}

        \vfill
        Departamento de Ingeniería Electrónica y Telecomunicaciones\\
        Universidad de Antioquia\\
        Medellín\\
        Marzo de 2021
                 
    \end{center}
\end{titlepage}

\tableofcontents

\section{Introduccion}
\noindent
Mediante este documento se presentará una actividad donde pondremos a prueba nuestra habilidad para realizar instrucciones con la menor ambiguedad posible, ya que serán interpretadas por una segunda persona y esta debe llegar a un resultado final que nosotros como programadores queremos lograr, las instrucciones mencionadas anteriormente seran listadas en el cuerpo de este informe y se pondrán en práctica con tres personas las cuales simplemente tendrán a la mano los materiales a utilizar y las instrucciones a ejecutar, esto será plasmado en un video para observar de que manera se desenvuelve la actividad.

\noindent
Cabe resaltar que en este documento se mostrarán dos imágenes con el estado A y el estado B. Pero, estas solo serán adjuntadas con el fin de darle una mayor explicación a nuestro trabajo, las personas con las que se pondrá a prueba la solución de la actividad no tendrán noción de estas imágenes, sino simplemente una hoja con las instrucciones propuestas. 

\section{Instrucciones} \label{contenido}
\noindent
Para el siguiente desafio tenemos las dos tarjetas y la hoja en un estado inicial(A), seguiremos el siguiente conjunto de instrucciones para llegar a un estado final(B). 
\begin{enumerate}

\item Levantar la hoja de papel y colocarla en otra parte del escritorio, de tal manera que la hoja deje de cubrir las tarjetas.

\item Tomar las dos tarjetas con una sola mano asegurandolas bien de tal forma que estas queden totalmente juntas. 

\item Con las tarjetas aseguradas en nuestra mano y ubicadas de forma vertical, las dirijimos hacia encima de la hoja de tal manera que el lado inferior de estas sea el  que toque la superficie de papel.

\item Luego de tener las tarjetas de manera vertical y sobre la hoja, con la misma mano que las tenemos aseguradas procedemos a ubicar la yema de nuestro dedo indice en la parte superior de las tarjetas, la yema del dedo pulgar la ubicamos en la parte lateral izquierda de estas. Por último, la respectiva yema del dedo medio y anular la ubicamos en la parte lateral derecha de las tarjetas.

\item Tras haber ubicado la yema de los dedos en su respectivo lugar, apoyamos nuestra muñeca sobre la hoja de papel, para tener un mayor equilibrio y para asegurar mejor la hoja sobre el escritorio.

\item Con la ayuda de los dedos que tenemos ubicados en los laterales empezamos a separar las dos tarjetas halando lentamente la primera tarjeta hacia nuestra mano, el dedo indice debe permanecer siempre en su posición y mantener unidas las dos tarjetas en sus bordes superiores, cuando veamos que las dos tarjetas están tomando la forma de una piramide frenamos el hale que estamos haciendo con nuestros dedos.

\item Luego de haber obtenido la piramide separamos el dedo pulgar, medio y anular de las tarjetas, sin dejar de apoyar el dedo indice que se encuentra encima de las dos tarjetas.

\item Por último, separamos lentamente el dedo indice y la muñeca de su posición con el objetivo de que las tarjetas queden en forma de piramide sin la ayuda de nuestros dedos, este paso debe realizarse cuidadosamente para que no se caigan las tarjetas y se desarme la piramide formada.  
    
\end{enumerate}

\noindent
A continuación se mostrarán unas imágenes donde podremos ver el estado inicial del desafio(estado A) y el punto final al que llegaremos después de haber realizado las anteriores instrucciones(estado B). 

\begin{figure}[h]
\includegraphics[width=6cm]{Estado A.jpeg}
\centering
\caption{Estado A}
\label{fig:Estado A}

\includegraphics[width=6cm]{Estado B.jpeg}
\caption{Estado B}
\label{fig:Estado B}
\end{figure}


\section{Conclusión} \label{conclulsion}
\noindent
Trás haber puesto en práctica las instrucciones anteriores con las tres personas, he podido concluir que el lenguaje a través del cual nos comunicamos los humanos mutuamente es muy ambiguo, y por lo más entendibles y precisos que queramos ser al momento de desarrollar pasos o simplemente enfrentarnos a algún tipo de problema, este se verá expuesto a fracasos o a malinterpretaciones, ya que las ideas no son procesadas por una segunda mente de la misma manera en la que nosotros estamos pensando, esto se debe a que todas las personas pensamos de forma totalmente distinta y lo que para cierto grupo de personas puede significar algo, para otra persona puede significar algo totalmente alejado de nuestro pensamiento.

\noindent
Finalmente, se puede ver entonces que el mayor problema radica en lo flexible que es nuestro lenguaje, a diferencia de cuando programamos una máquina, ya que los lenguajes de programación aqui se encuentran libre de cualquier tipo de ambiguedad y toda instrucción a ejecutar debe ser totalmente precisa para obtener resultados favorables. 

\end{document}
