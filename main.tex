\documentclass{article}
\usepackage[utf8]{inputenc}
\usepackage[spanish]{babel}
\usepackage{listings}
\usepackage{graphicx}
\graphicspath{ {images/} }
\usepackage{cite}

\begin{document}

\begin{titlepage}
    \begin{center}
        \vspace*{1cm}
            
        \Huge
        \textbf{Actividad N°1}
            
        \vspace{0.5cm}
        \LARGE
        Parcial 1 - Calistenia (15 porciento)
            
        \vspace{1.5cm}
            
        \textbf{Johan David Rojas Martinez}
            
        \vfill
            
        \vspace{0.8cm}
       
        \Large
\begin{figure}[h]
\includegraphics[width=4cm]{logoudea.png}
\centering
\end{figure}

        \vfill
        Despartamento de Ingeniería Electrónica y Telecomunicaciones\\
        Universidad de Antioquia\\
        Medellín\\
        Marzo de 2021
                 
    \end{center}
\end{titlepage}

\tableofcontents

\section{Introduccion}
\noindent
Mediante este documento se presentara una actividad donde se pondrá a prueba nuestra habilidad para realizar instrucciones con la menor ambiguedad posible, ya que serán interpretadas por una segunda persona y queremos que esta llegue al resultado final que nosotros como programadores queremos lograr, las instrucciones mencionadas anteriormente seran listadas en el cuerpo de este informe y se pondrán en práctica con tres personas las cuales simplemente tendran a la mano los materiales a utilizar y las instrucciones a ejecutar, esto será plasmado en un video para observar de que manera se desenvuelve la actividad. 

\section{Instrucciones} \label{contenido}
\noindent
Para el siguiente desafio tenemos las dos tarjetas y la hoja en un estado inicial(A), seguiremos el siguiente conjunto de instrucciones para llegar a un estado final(B). 
\begin{enumerate}

\item Levantar la hoja de papel y colocarla en otra parte del escritorio, de tal manera que la hoja deje de cubrir las tarjetas.

\item Tomar las dos tarjetas con una sola mano asegurandolas bien de tal forma que estas queden totalmente juntas y alineadas. 

\item Con las tarjetas aseguradas en nuestra mano las dirijimos hacia encima de la hoja de tal manera que la parte más corta de las tarjetas sean las que toquen la superficie de papel.

\item Luego de tener las tarjetas de manera vertical sobre la hoja, con la misma mano que las tenemos aseguradas procedemos a ubicar nuestro dedo indice en la parte superior de las tarjetas, el dedo pulgar lo ubicamos en la parte lateral izquierda de estas. Por último, el dedo medio y anular lo ubicamos en la parte lateral derecha de las tarjetas.

\item Tras haber ubicado la yema de los dedos en su respectivo lugar, apoyamos nuestra muñeca sobre la hoja de papel, para tener un mayor equilibrio y para asegurar mejor la hoja sobre el escritorio.

\itme Separamos las dos tarjetas teniendo en cuenta que el dedo indice no lo debemos separar de su posición.

\item Con la ayuda de los dedos que tenemos ubicados en los laterales empezamos a separar las dos tarjetas halando lentamente la primera tarjeta hacia nuestra mano, el dedo indice debe permanecer siempre en su posicion y mantener unidas las dos tarjetas en sus bordes superiores, cuando veamos que las dos tarjetas estan tomando la forma de una piramide frenamos el hale que estamos haciendo con nuestros dedos.

\item Luego de haber obtenido la piramide separamos el dedo pulgar, medio y anular de las tarjetas, sin dejar de apoyar el dedo indice que se encuentra encima de las dos tarjetas.

\item Por último, separamos lentamente el dedo indice y la muñeca de su posición con el objetivo de que las tarjetas queden en forma de piramide sin la ayuda de nuestros dedos, este paso debe realizarse cuidadosamente para que no se caigan las tarjetas y se desarme la piramide formada.  
    
\end{enumerate}


\section{Conclusión} \label{conclulsion}
\noindent
En este espacio se realizara lo que seria la conclusion de este trabajo.

\end{document}
